\documentclass[12pt]{report}
\usepackage{natbib}
\usepackage{url}
\usepackage[utf8x]{inputenc}
\usepackage{amsmath}
\usepackage{graphicx}
\graphicspath{{images/}}
\usepackage{parskip}
\usepackage{fancyhdr}
\usepackage{vmargin}
\usepackage{lmodern}
\usepackage[T1]{fontenc}
\usepackage[super]{nth}
\usepackage{rotating}
\usepackage{tikz}
\usepackage{lscape}
\usepackage{geometry}
\usepackage{caption}
\usepackage{makecell}
\usepackage{subcaption}
\usepackage[page]{appendix}
\usepackage{minted}
\usepackage{verbatim}
\setcounter{secnumdepth}{4}

\usepackage{hyperref}
\usepackage[xindy]{glossaries} 
\hypersetup{
    colorlinks=true, % false: boxed links; true: colored links
    linkcolor=black, % color of internal links
    citecolor=black, % color of links to bibliography
    filecolor=black, % color of file links
    urlcolor=black % color of external links
}
%\renewcommand*{\glstextformat}[1]{\textcolor{black}{#1}}
\renewcommand{\glstextformat}[1]{\textit{#1}}


\setmarginsrb{3 cm}{2.5 cm}{3 cm}{2.5 cm}{1 cm}{1.5 cm}{1 cm}{1.5 cm}
\title{ML Cheatsheet}								% Title
\author{Axel Mendoza}								% Author
\date{November 17, 2020}											% Date

\makeatletter
\let\thetitle\@title
\let\theauthor\@author
\let\thedate\@date
\makeatother

\pagestyle{fancy}
\fancyhf{}
\rhead{\theauthor}
\lhead{\thetitle}
\cfoot{\thepage}

\let\origdoublepage\cleardoublepage
\newcommand{\clearemptydoublepage}{%
  \clearpage
  {\pagestyle{empty}\origdoublepage}%
}

\input{glossary}
\makeglossaries
\begin{document}

%%%%%%%%%%%%%%%%%%%%%%%%%%%%%%%%%%%%%%%%%%%%%%%%%%%%%%%%%%%%%%%%%%%%%%%%%%%%%%%%%%%%%%%%%

\begin{titlepage}
	\centering
    \includegraphics[width=0.51\linewidth]{images/epita_logo.png}\\[2.0 cm]	% University Logo
    \textsc{\Large Introducing}\\[0.4 cm]
    %\textsc{{\large Confidential Report}}\\[0.2 cm]				% Course Name
	\rule{\linewidth}{0.2 mm} \\[0.4 cm]
	{ \huge \bfseries \thetitle}\\
	\rule{\linewidth}{0.2 mm} \\[0.6 cm]
	
	\begin{minipage}{0.4\textwidth}
		\begin{flushleft} \large
			\emph{Author:}\\
			\theauthor
			\end{flushleft}
			\end{minipage}~
			\begin{minipage}{0.4\textwidth}
			\begin{flushright} \large
			\emph{College:} \\
            Epita 2018									% Your Student Number
		\end{flushright}
		
	\end{minipage}\\[0.6 cm]
	
	{\large \thedate}\\[2 cm]
    \includegraphics[width=0.45\linewidth]{images/58065.jpg}\\[2.0 cm]	% University Logo
    %\includegraphics[width=0.45\linewidth]{engie_lab.png}	% University Logo
	\vfill
	
\end{titlepage}

%%%%%%%%%%%%%%%%%%%%%%%%%%%%%%%%%%%%%%%%%%%%%%%%%%%%%%%%%%%%%%%%%%%%%%%%%%%%%%%%%%%%%%%%%
{
\tableofcontents
}
\pagebreak

%%%%%%%%%%%%%%%%%%%%%%%%%%%%%%%%%%%%%%%%%%%%%%%%%%%%%%%%%%%%%%%%%%%%%%%%%%%%%%%%%%%%%%%%%
\chapter{Exploratory data analysis}
    \section{Section 1}
    
\chapter{Machine Learning}
\chapter{Deep Learning}

    
        %A figure
        %\begin{figure}[ht]
        %    \centering
        %    \includegraphics[width=1\textwidth]{engie_lab.png}
        %    \caption{Example of human \gls{re-id} data labeling task}
        %\end{figure}
        %
        %Formula in line $\displaystyle \mathbf{w} \cdot \mathbf{x}$.
        %Formula with caption
        
        %\begin{equation}
        %H[n]=\begin{cases} 0, & n < 0, \\ 1, & n \ge 0, \end{cases}
        %\end{equation}
        %\subsection{Sub Section 1}\label{subsec1}
        %    %A glossary entry \gls{MLP}.
        %    %Two figures side by side.
        %    \begin{figure}[ht]
        %    \centering
        %    \begin{minipage}{.5\textwidth}
        %        \centering
        %        \includegraphics[width=1\textwidth]{epita_logo.png}
        %        \captionof{figure}{Convolution of stride 1}
        %        \label{fig:convstrideone}
        %    \end{minipage}%
        %    \begin{minipage}{.5\textwidth}
        %        \centering
        %        \includegraphics[width=1\textwidth]{engie_lab.png}
        %        \captionof{figure}{Convolution of stride 2}
        %        \label{fig:convstridetwo}
        %    \end{minipage}
        %    \end{figure}
        %    
        %    Itemize:
        %    \begin{itemize}
        %        \item Convolution layer for analyzing pattern in the image.
        %        \item Max Pooling layer to reduce the dimension of the input.
        %        \item Fully Connected layer to detect global configurations of the features extracted by the convolution layers.
        %        \item Softmax layer for classification.
        %    \end{itemize}
        %    
        %    Referencing the above Figure~\ref{fig:tripletnetwork}.
        %    \begin{figure}[ht]
        %        \centering
        %        \includegraphics[width=0.5\textwidth]{epita_logo.png}
        %        \caption{Offline Triplet Mining Schema.}
        %        \label{fig:tripletnetwork}
        %    \end{figure}
        %    
        %    Complex formula:
        %    \begin{equation}
        %    %\scalebox{1.00}{
        %        \begin{gathered}
        %            \displaystyle{w _ { i } ^ { + } = \left( d \left( \boldsymbol { f } _ { a }, \boldsymbol { f } _ { i } \right) + 1\right) ^ { \alpha }
        %            \quad ~~~~\text { if } \boldsymbol { f } _ { i } \in \boldsymbol { S } _ { a } ^ { + },} \\
        %            \displaystyle{w _ { j } ^ { - } = \left( d \left( \boldsymbol { f } _ { a } , \boldsymbol { f } _ { j } \right) + 1 \right) ^ { - 2 \alpha } \quad \text { if } \boldsymbol { f } _ { j } \in \boldsymbol { S } _ { a } ^ { - }}
        %        \end{gathered}%}
        %        \label{eq:batchhard}
        %    \end{equation}
        %    
        %    A nice table:
        %    \begin{figure}[ht]
        %        \centering
        %        \noindent\resizebox{\textwidth}{!}{%
        %        \begin{tabular}{|| l | c | c | c | l | c ||}
        %            \hline
        %            Method & mAP (\%) & Rank-1 & Rank-5 & Model & Year\\
        %            \hline
        %            \hline
        %            FACT+Plate+SNN+STR  & 27.77 & 61.44 & 78.78 & None & 2015 \\
        %            \hline
        %            OIFE-4-Views & 48.00 & 89.43 & ? & Custom & 2017 \\
        %            \hline
        %            VAMI & 50.13 & 77.03 & 90.82 & Custom & 2018 \\
        %            \hline
        %            Siamese-CNN+Path-LSTM  & 58.27 & 83.49 & 90.04 & RN50+LSTM & 2017 \\
        %            \hline
        %            Semihard+softmax+AIC  & 57.43 & 86.29 & 94.39 & RN50 & 2018\\
        %            \hline
        %            GAN+LSRO & 58.23 & 87.70 & 93.92 & RN50+DCGAN & 2018 \\
        %            \hline
        %            RAM-Multi-Learners  & 61.50 & 88.60 & 94.00 & VGG19x4 & 2018 \\
        %            \Xhline{4\arrayrulewidth}
        %            Baseline  ~~~~~~~~~~~~~~~~~($\boldsymbol{Ours}$)& 53.78 & 81.46 & 91.47 & RN50 & 2018\\
        %            \hline
        %            Hap2s+softmax \cite{spacetimeprior}+ ($\boldsymbol{Ours}$) & 56.85 & 84.80 & 92.43 & RN50 & 2018\\
        %            \hline
        %        \end{tabular}}
        %        \captionof{table}{Comparison with the related works on VeRi-776 }\label{fig:scores}
        %    \end{figure}
% - Limitation
\printglossaries

\bibliographystyle{plain}
\bibliography{biblist}
\end{document}